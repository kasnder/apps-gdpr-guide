\documentclass[
	12pt,
	a4paper,
	%oneside
	]{scrartcl}

\usepackage[utf8]{inputenc}
\usepackage[T1]{fontenc}
\usepackage{lmodern}
\usepackage[autostyle=true]{csquotes} % Quotes
\usepackage[UKenglish]{babel} % Language support
\usepackage{url}
\usepackage{microtype} % improve text
\usepackage[splitrule,multiple]{footmisc} % prevent footnotes from breaking
\usepackage[pdftex,
pdfauthor={Konrad Kollnig},
pdftitle={An App Developer's Guide to 
GDPR},hidelinks,hyperfootnotes=false,bookmarksopen=true]{hyperref}
 
\usepackage{longtable}
\usepackage{booktabs}

%%% Some options
\setlength{\parindent}{0pt}
\setlength{\parskip}{10pt}

%%% Slightly larger line spacing
\usepackage{setspace}
\linespread{1.5}

\begin{document}
	\title{An App Developer's Guide to GDPR}
	\author{Konrad Kollnig}
	\date{Version: \today}
	\maketitle
	
	\begin{center}
		\emph{No legal advice, only an app developer’s attempt to make 
		data protection more understandable.}
	\end{center}
	
	If you have app users from the European Union, you are responsible for
	personal data collected through your app. Personal data is data 
	relating
	to individuals. This may include device data, pseudonyms, user
	identifiers, advertising identifiers, (dynamic) IP addresses, and
	postcodes, especially in combination with other data. For these 
	reasons,
	it is usually not possible to make personal data non-personal.
	
	You are also responsible for personal data collected from your app for
	third-party services, such as advertising, analytics, or crash 
	reporting
	services.
	
	\textbf{Risk evaluation and documentation.} GDPR acknowledges that 
	there
	will never be full protection of personal data. Instead, it encourages 
	a
	risk-based approach, that is, seriously analysing the possible risks to
	data protection and taking appropriate data protection measures. If you
	can prove that you took all appropriate measures, there is no need to 
	be
	overly afraid of high fines.
	
	Make sure that you can provide such proof, by \emph{documenting all 
	data
	protection considerations, decisions, and actions}.
	
	\textbf{Reasonable data collection.} You and your third-parties may 
	only
	collect personal data reasonably, that is, only for the
	purposes stated in your privacy policy (purpose limitation) and 
	restricted to what is necessary for the stated purposes (data 
	minimisation).
	
	Furthermore:
	\begin{itemize}
		\item
		\textit{iOS:} According the Apple's terms, you should ask
		for user consent, before you or your third-parties collect 
		\emph{any data}, no matter if personal and non-personal.\\	
		\textit{Android:}  According
		to Google's terms, if you process sensitive data (e.g. 
		health-related), 
		or process data in unexpected ways, do tell the user in a clear 
		manner 
		and ask for his \textit{consent} (no pre-ticked boxes allowed).
		\item At best, use at most one third-party service for one 
		purpose, 
		that is, at most one advertising, analytics, and crash reporting 
		service.
		\item Check with every app release, if you can reduce data 
		collection or remove any third-party services.
		\item Verify the default settings of your third-party services 
		(on-device and server-wise), since third-parties have an interest 
		in 
		collecting ever more data.
		Only activate third-party services, once user consent is 
		established. More information can be
		found in the Appendix below.
		\item If your app is aimed at \textit{children}, do not employ any 
		third-party services. It's not good practice, and a
		violation of Apple's terms.
		\item If possible, use libraries that make their source code available. Otherwise, you have no means to verify the underlying data practices.
	\end{itemize}
	
	\textbf{Always provide a privacy policy.}
	Provide a privacy policy on the app store and within the app.
	You may want to use one of the privacy policy generators, such as 
	\url{iubenda.com}.
	Make sure it discloses the data collection of you and your 
	third-parties 
	adequately.
	
	\textbf{Handling user requests.}
	The GDPR entitles users to manage
	(e.g.~access, delete, correct) any data about them. You can implement
	these user rights directly in software, which would show your efforts
	towards GDPR compliance. Yet, taking requests via email seriously is
	just as fine. You have one month to respond to user requests. This
	response may either address the request, or, for complex user requests,
	request an extension for a further 2 months.
	
	\textbf{Security measures and data breaches.}
	Take the standard measures for security, such as HTTPS communications, 
	salted passwords, validation of user inputs.
	Apple\footnote{\url{https://developer.apple.com/documentation/security}}\footnote{\url{https://developer.apple.com/library/archive/documentation/Security/Conceptual/SecureCodingGuide}}
	 and 
	Google\footnote{\url{https://developer.android.com/training/articles/security-tips}}
	 provide comprehensive guidance on this.
	Try to remove identifiable information whenever possible, through 
	pseudonymisation or anonymisation.
	If you experience a \textit{personal data breach}, you must notify the 
	data protection 
	authority\footnote{\url{https://edpb.europa.eu/about-edpb/board/members}}
	 
	within 72 hours, plus the individuals in case of high risk.
	
	\textbf{Consent for third-party services.}
	If you use third-party services, the user must be asked for consent in 
	almost all circumstances.
	This consent must be sought before the third-party service is 
	activated 
	and begins to share data.
	%The usage of third-party services, such as Google Analytics or 
	%Facebook 
	%Advertising, centralises data about a large number of users.
	%Whilst there is no principle reason against using these services, you 
	%must assess the risks to the individual carefully and bear 
	%responsibility 
	%for taking all reasonable steps to minimise these risks.
	%You should be aware that third-parties profit from amassing data, and 
	%carefully assess what data is shared by the third-party.
	%Make sure that you activate a third-party SDK, only when user consent 
	%is 
	%granted.
	%Include detail about the third-party services in your app's privacy 
	%policy, just as you would about your own app; the policies of the 
	%third-party service tell you what to do.
	Beyond consent, the Appendix
	provides 
	more detail on the correct implementation of the most widely used 
	third-party services.
	
	\textbf{Closing remarks.}
	By implementing these measures, you should come an important step 
	closer 
	to compliance with GDPR.
	Additionally, you should consult the guidelines of an EU data 
	protection 
	authority.
	The British Data Protection Authority, called ICO, provides excellent 
	guidance\footnote{\url{https://ico.org.uk/for-organisations/}} on data 
	protection.
	
	\pagebreak
	
	\section*{Appendix: Using Third-Party Services}
	Implementation guidance for the most commonly used third-party 
	services, as well as links to their GDPR guidelines.
	
	\begin{footnotesize}
		\begin{longtable}{lp{.65\textwidth}}
			%\caption{Implementation guidance for the most commonly used 
			%third-party services, as well as links to their GDPR 
			%guidelines.}\label{tab:implementation_guidance_trackers} \\
			\toprule
			Service & Implementation Notes \\ 
			\midrule 
			Adjust & Once the Adjust SDK is enabled in your app, data 
			sharing 
			takes place, notably of device events.
			User consent should be established before enabling this SDK.
			It stands out that Adjust integrates the GDPR \textit{right to 
			deletion} into their SDK. This could be implemented in your 
			app, 
			to show your efforts to comply with GDPR. \newline 
			\textbf{More 
			info:} \url{https://github.com/adjust/sdks} \\
			\midrule 
			AppLovin & For EU users, AppLovin requires consent to be 
			passed on 
			programmatically.
			If consent is given, the Advertising ID and IP address will be 
			sent to advertising partners, otherwise only a country code.
			Once loaded at runtime,
			AppLovin automatically receives the information that the app 
			was 
			installed. \newline \textbf{More info:} 
			\url{https://www.applovin.com/gdprfaqs/} \\
			\midrule 
			AppsFlyer & The service collects the Advertising ID and a 
			unique 
			AppsFlyer user ID from the first app start.
			User consent should be established before activating this 
			service.
			If the Advertising ID cannot be accessed, permanent 
			identifiers, 
			notably the device's IMEI, are shared with AppsFlyer, unless 
			programmatically disabled. Such permanent identifiers are 
			highly 
			critical from a data protection standpoint.
			This practice should be communicated transparently to the 
			user, if 
			not disabled. \newline
			\textbf{More info:} 
			\url{https://support.appsflyer.com/hc/en-us/articles/360001422989}.
			 \\
			\midrule 
			Facebook SDK & From the first app start, the Facebook SDK 
			collects 
			device information and events (app installation, app start, 
			in-app 
			purchases), unless programmatically disabled.
			User consent should be established before activating this SDK.
			Facebook serves no advertising, if the user limits 
			interest-based 
			ads from the device settings. \newline \textbf{More info:} 
			\url{https://developers.facebook.com/docs/app-events/best-practices/gdpr-compliance}
			 \\
			\midrule 
			Flurry & For ads, this service provides a complicated 
			mechanism to 
			establish a user consent.
			Since legally required for many advertising services,
			you may want to consider easier, alternative approaches to 
			establish valid user consent.
			Unless programmatically disabled, the user location is 
			collected 
			for analytics purposes, if the app has the permission to 
			retrieve 
			such.
			This is highly invasive and may violate GDPR.
			At very least, this practice should be disclosed to the user 
			transparently, if not disabled.
			Generally, user consent should be established before 
			activating 
			this service. \newline \textbf{More info (Analytics):} 
			\url{https://developer.yahoo.com/flurry/docs/analytics/gdpr/summary}
			 \newline \textbf{More info (Ads):} 
			\url{https://developer.yahoo.com/flurry/docs/publisher/gdpr/} 
			\\
			\midrule 
			Google AdMob & This service serves personalised advertising by 
			default, violating Google's policies if used in the EU.
			This must be changed by the developer, such that user consent 
			is 
			established prior to serving personalised ads.
			AdMob shares device statistics and events with Google from the 
			first app start, unless programmatically changed.
			User consent should be established before activating this 
			service. 
			\newline \textbf{More info:} 
			\url{https://developers.google.com/admob/android/eu-consent\#forward_consent_to_the_google_mobile_ads_sdk}.
			 \\
			\midrule 
			Google Analytics & User opt-out and IP anonymisation are 
			supported 
			programmatically and their implementation should be 
			considered. 
			User consent should be established before using this service. 
			\newline \textbf{More info:}
			\url{https://developers.google.com/analytics/devguides/collection/android/v4/advanced}
			 \\
			\midrule
			Google Crashlytics & This service shares crash reports with 
			Google from the first app start, unless changed by the 
			developer. 
			User consent should be established before activating this 
			service. 
			\newline \textbf{More info:}
			\url{https://firebase.google.com/docs/crashlytics/customize-crash-reports\#enable_opt-in_reporting}
			 \\
			\midrule 
			Google DoubleClick & This service serves personalised 
			advertising 
			by default,
			violating Google's policies if used in the EU.
			User consent should be established before activating this 
			service. 
			\newline \textbf{More info:} 
			\url{https://developers.google.com/ad-manager/mobile-ads-sdk/android/eu-consent\#forward_consent_to_the_google_mobile_ads_sdk}.
			 \\ 
			\midrule 
			Google Firebase Analytics & This service collects device 
			statistics from the first app start, unless changed by the 
			developer.
			The collected data includes the Google Advertising ID, unless 
			programmatically disabled, and may be used for advertising 
			purposes under certain circumstances. User consent should be 
			established before activating this service. \newline 
			\textbf{More 
			info:} 
			\url{https://firebase.google.com/docs/analytics/configure-data-collection}
			 \\
			\midrule 
			Inmobi & The Inmobi SDK only collects personal data, if you 
			explicitly indicate to the SDK that user consent was 
			established.
			If no consent is given, unpersonalised ads are shown to the 
			user.
			Inmobi encourages you to provide data about location and 
			demographics for higher revenue, if you programmatically pass 
			on 
			this information. Such sensitive data collection should be 
			transparently disclosed to the user, if not refrained from. 
			\newline \textbf{More info:} 
			\url{https://support.inmobi.com/monetize/faqs/gdpr-guide-for-publishers/}
			   \\
			\midrule 
			MoPub & For increased advertising revenue, MoPub shares data 
			with 
			two other services, IAS and Moat, unless programmatically 
			disabled.
			These services must be transparently communicated to the user, 
			if 
			not disabled.
			User consent should be established before activating this 
			service. 
			\newline \textbf{More info:} 
			\url{https://developers.mopub.com/publishers/best-practices/gdpr-guide/}
			 \\
			\midrule 
			Unity Ads & Unity automatically asks for user consent, unless 
			a 
			special arrangement is reached with Unity.
			Personal data is only collected if the user consents.
			When ads are served, Unity provides the user with a 
			\enquote{privacy icon}, to change his opt-out setting.
			If the user opts-out, all collected data is deleted. \newline 
			\textbf{More info:} \url{https://unity3d.com/de/legal/gdpr} \\
			\bottomrule 
		\end{longtable}
	\end{footnotesize}

\end{document} 
